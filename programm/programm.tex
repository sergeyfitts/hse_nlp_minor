\documentclass[a4paper,11pt]{article}
%pdflatex
\usepackage{cmap}					% поиск в PDF
\usepackage{mathtext} 				% русские буквы в формулах
\usepackage[T2A]{fontenc}			% кодировка
\usepackage[utf8]{inputenc}			% кодировка исходного текста
\usepackage[russian]{babel}	
\usepackage{indentfirst}
\frenchspacing

%математика
\usepackage{amsmath,amsfonts,amssymb,amsthm,mathtools} % AMS
\usepackage{icomma}
\newcommand*{\hm}[1]{#1\nobreak\discretionary{}
	{\hbox{$\mathsurround=0pt #1$}}{}}

\usepackage{extsizes} % Возможность сделать экстрашрифт
\usepackage{geometry} 
\geometry{top=30mm}
\geometry{bottom=40mm}
\geometry{left=30mm}
\geometry{right=20mm}

\usepackage{setspace}
\onehalfspacing

\usepackage{hyperref}
\usepackage[usenames,dvipsnames,svgnames,table,rgb]{xcolor}
\hypersetup{				% Гиперссылки
	unicode=true,           % русские буквы в раздела PDF
	pdftitle={Заголовок},   % Заголовок
	pdfauthor={Автор},      % Автор
	pdfsubject={Тема},      % Тема
	pdfcreator={Создатель}, % Создатель
	pdfproducer={Производитель}, % Производитель
	pdfkeywords={keyword1} {key2} {key3}, % Ключевые слова
	colorlinks=true,       	% false: ссылки в рамках; true: цветные ссылки
	linkcolor=black,          % внутренние ссылки
	citecolor=black,        % на библиографию
	filecolor=magenta,      % на файлы
	urlcolor=blue           % на URL
}
\author{Чувакин С. Н.}
\title{Информационный поиск и обработка текстов на естественном языке}
\date{\today}

\let\endtitlepage\relax % убрать pagebreak после titlepage
\usepackage{enumitem} % цветные itemize (xcolor)
\setcounter{secnumdepth}{0} % убрать нумерацию секций

\usepackage{csquotes}
\usepackage[style=apa, maxcitenames=2,backend=biber,sorting=nty, bibencoding = utf8]{biblatex}
\addbibresource{bib.bib}

\begin{document}
	
\maketitle
 
\subsubsection{10.09.2019}
\begin{itemize}
\item Знакомство + о курсе + организационные моменты + учесть пожелания.  
\item Введение в NLP: что это такое, зачем business (and not only) application. Виды задач и их решения. 
\end{itemize}

\subsubsection{17.09.2019}
\begin{itemize}
	\item Простая векторизация текстов. BOW, n-grams, tf-idf. Лемматизация, стеммиг. 
	\item Практика на питоне. Рассмотрим gensim и sklearn.  
\end{itemize}
\subsubsection{24.09.2019}
\begin{itemize}
	\item Откуда брать данные? Готовые датасеты vs. самостоятельный сбор.
	\item Работа со строками в питоне. Regex, bs4, selenium. 
\end{itemize}
\subsubsection{01.10.2019}
\begin{itemize}
	\item Морфоанализ,  POS tagging. 
	\item Практика на питоне 
\end{itemize}
\subsubsection{8.10.2019}
\begin{itemize}
	\item Введение в классификация текстов. Topic modeling,sentiment analysis, простые рубрикаторы.  
	\item Практика на питоне. LSI.
	\item ДЗ: LSI своими руками, без готовых решений на gensim, sklearn. 
\end{itemize}
\subsubsection{15.10.2019}
\begin{itemize}
	\item Классификация текстов - вероятностные подходы LDA, RP (Random Projections), HDP (Hierarchical Dirichlet Process). Метрики качества. 
	\item Практика с gensim. 
\end{itemize}
\subsubsection{22.10.2019}
\begin{itemize}
	\item Sentiment analysis.
	\item практика на питоне. 
\end{itemize}
\subsubsection{27.10.2019 Сессия}
\begin{itemize}
	\item  Задание на скачку постов с кинопоиска (или с роспотребнадзора, который реализован на AJAX). Их классификая и выделение тем. 
	\item 
\end{itemize}
\subsubsection{05.11.2019}
\begin{itemize}
	\item Текстовая схожесть. word2vec, seq2vec, doc2vec
	\item Практика на питоне. 
\end{itemize}
\subsubsection{12.11.2019}
\begin{itemize}
	\item Извлечене информации из текста. NER, dependency matrix.
	\item Анализ готовых решений и их имплементация. 
\end{itemize}
\subsubsection{19.11.2019}
\begin{itemize}
	\item Simple-chatbots. Conversational vs. goal oriented.
	\item Пробуем написать своего чат бота.
\end{itemize}
\subsubsection{26.11.2019}
\begin{itemize}
	\item Машинный перевод.
	\item Машинный перевод. Encoders-decoders. 
\end{itemize}
\subsubsection{03.12.2019}
\begin{itemize}
	\item Глубокое обучение.
	\item Обзор современных методов и решений.  
\end{itemize}

\subsubsection{10.12.2019}
\begin{itemize}
	\item Трансформеры.
\item Обзор современных решений.  
\end{itemize}
\subsubsection{17.12.2019}
\begin{itemize}
	\item Подготовка к финальному проекту. 
	\item  
\end{itemize}
\end{document}
